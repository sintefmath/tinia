    \pgfmathsetmacro\s{1}
    \pgfmathsetmacro\posx{0.5*(\s-1) + 2}
    \pgfmathsetmacro\posy{3.5 - 0.6*(\s-1)}
    \pgfmathsetmacro\posxRot{rotx(\posx-\cx, \posy-\cy, 0)+\cx}
    \pgfmathsetmacro\posyRot{roty(\posx-\cx, \posy-\cy, 0)+\cy}
    \coordinate (left) at ( \posxRot, \posyRot );
    
    \pgfmathsetmacro\s{\redpixels - 1}
    \pgfmathsetmacro\posx{0.5*(\s-1) + 2}
    \pgfmathsetmacro\posy{3.5 - 0.6*(\s-1)}
    \pgfmathsetmacro\posxRot{rotx(\posx-\cx, \posy-\cy, 0)+\cx}
    \pgfmathsetmacro\posyRot{roty(\posx-\cx, \posy-\cy, 0)+\cy}
    \coordinate (middle) at ( \posxRot, \posyRot );
    
    % The offset necessary for the red and green lines to meet properly
    \pgfmathsetmacro\redgreenintersection{3.5 -(0.6+0.2)*(\s-1)}
    
    \pgfmathsetmacro\s{\pixels}
    \pgfmathsetmacro\posx{0.5*(\s-1) + 2}
    \pgfmathsetmacro\posy{0.2*(\s-1) + \redgreenintersection}
    \pgfmathsetmacro\posxRot{rotx(\posx-\cx, \posy-\cy, 0)+\cx}
    \pgfmathsetmacro\posyRot{roty(\posx-\cx, \posy-\cy, 0)+\cy}
    \coordinate (right) at ( \posxRot, \posyRot );
    
    % The "server-rendered model"
    \fill[left color=\leftcolLToUse, right color=\leftcolRToUse, draw=none]
      (left) -- (middle) -- +(0, 5*\linewidth) -- ($ (left) + (0, 5*\linewidth) $) -- cycle;
    % Why is the factor 3 needed here?!?!
    % And why is there a (very thin) border?
    % And why is it not equally thick everywhere?
    \draw[\rightcolToUse, line width=\linewidth] (middle) -- (right);

    \ifthenelse{ \withSplats = 1 } {
      \foreach \s in {1, ..., \pixels} {
        \ifthenelse{\s < \redpixels} {
          \pgfmathsetmacro\posx{0.5*(\s-1) + 2}
          \pgfmathsetmacro\posy{3.5 - 0.6*(\s-1)}
          \pgfmathsetmacro\posxRot{rotx(\posx-\cx, \posy-\cy, \theta)+\cx}
          \pgfmathsetmacro\posyRot{roty(\posx-\cx, \posy-\cy, \theta)+\cy}
          
          % Screen-space-sized splats
          \pgfmathsetmacro\posx{0.5*(\s-1-1) + 2}
          \pgfmathsetmacro\posy{3.5 - 0.6*(\s-1-1)}
          \pgfmathsetmacro\posxRotL{rotx(\posx-\cx, \posy-\cy, \theta)+\cx}
          \pgfmathsetmacro\posyRotL{roty(\posx-\cx, \posy-\cy, \theta)+\cy}
          \pgfmathsetmacro\posx{0.5*(\s-1) + 2}
          \pgfmathsetmacro\posy{3.5 - 0.6*(\s-1)}
          \pgfmathsetmacro\posxRotR{rotx(\posx-\cx, \posy-\cy, \theta)+\cx}
          \pgfmathsetmacro\posyRotR{roty(\posx-\cx, \posy-\cy, \theta)+\cy}
          \pgfmathsetmacro\halfWidthToUse{0.5*(\posxRotR-\posxRotL)}
          
          % Overriding with constant width for one of the sub-figures
          \ifthenelse{ \withConstantWidthSplats = 1 } {
            \pgfmathsetmacro\halfWidthToUse{\halfpixelwidth}
          }{}

          \ifthenelse{ \withLargeSplats = 1 } {
            % Varying color here on the left side
            \pgfmathsetmacro\tl{100*(\s-1)/\redpixels}
            \pgfmathsetmacro\ul{100-\tl)}
            \pgfmathsetmacro\tr{100*(\s  )/\redpixels}
            \pgfmathsetmacro\ur{100-\tr}
            \ifthenelse{ \withFragDepth = 0 } {
              \draw[left color=red!\tl!blue!\ul, right color=red!\tr!blue!\ur]
                (\posxRot-\halfWidthToUse, \posyRot) rectangle +(2*\halfWidthToUse, 0.2);
            }{
              % Tror kanskje at problemet med at ting kommer for langt til venstre her, skyldes at vi ser paa 
              % linjestykker fra s-1-1 til s-1. Burde muligens vaert fra s-1-0.5 til s-1+0.5?!
              % Replace with rotated rectangles, will probably look better...
              \draw[left color=red!\tl!blue!\ul, right color=red!\tr!blue!\ur]
                (\posxRotL, \posyRotL) -- (\posxRotR, \posyRotR) -- +(0, 5*\linewidth) -- 
                ($ (\posxRotL, \posyRotL) + (0, 5*\linewidth) $) -- cycle;
            }
          }{
            % Constant color, just a dot
            \pgfmathsetmacro\t{100*\s/\redpixels}
            \pgfmathsetmacro\u{100*(1.0-\s/\redpixels)}
            \draw[fill=red!\t!blue!\u] (\posxRot, \posyRot) circle(0.15);
          }
        } {
          \pgfmathsetmacro\posx{0.5*(\s-1) + 2}
          \pgfmathsetmacro\posy{0.2*(\s-1) + \redgreenintersection}
          \pgfmathsetmacro\posxRot{rotx(\posx-\cx, \posy-\cy, \theta)+\cx}
          \pgfmathsetmacro\posyRot{roty(\posx-\cx, \posy-\cy, \theta)+\cy}
          
          % Screen-space-sized splats
          \pgfmathsetmacro\posx{0.5*(\s-1-1) + 2}
          \pgfmathsetmacro\posy{0.2*(\s-1-1) + \redgreenintersection}
          \pgfmathsetmacro\posxRotL{rotx(\posx-\cx, \posy-\cy, \theta)+\cx}
          \pgfmathsetmacro\posyRotL{roty(\posx-\cx, \posy-\cy, \theta)+\cy}
          \pgfmathsetmacro\posx{0.5*(\s-1) + 2}
          \pgfmathsetmacro\posy{0.2*(\s-1) + \redgreenintersection}
          \pgfmathsetmacro\posxRotR{rotx(\posx-\cx, \posy-\cy, \theta)+\cx}
          \pgfmathsetmacro\posyRotR{roty(\posx-\cx, \posy-\cy, \theta)+\cy}
          \pgfmathsetmacro\halfWidthToUse{0.5*(\posxRotR-\posxRotL)}
          
          \ifthenelse{ \withConstantWidthSplats = 1 } {
            % Overriding with constant width for one of the sub-figures
            \pgfmathsetmacro\halfWidthToUse{\halfpixelwidth}
          }{}

          \ifthenelse{ \withLargeSplats = 1 } {
            \ifthenelse{ \withFragDepth = 0 } {
              \draw[fill=\rightcolToUse]
                (\posxRot-\halfWidthToUse, \posyRot) rectangle +(2*\halfWidthToUse, 0.2);
            }{
              % \draw[fill=\rightcolToUse, line width=1.5pt] (\posxRotL, \posyRotL) -- (\posxRotR, \posyRotR);
              % Tror kanskje at problemet med at ting kommer for langt til venstre her, skyldes at vi ser paa 
              % linjestykker fra s-1-1 til s-1. Burde muligens vaert fra s-1-0.5 til s-1+0.5?!
              % Replace with rotated rectangles, will probably look better...
              \draw[fill=\rightcolToUse]
                (\posxRotL, \posyRotL) -- (\posxRotR, \posyRotR) -- +(0, 5*\linewidth) -- 
                ($ (\posxRotL, \posyRotL) + (0, 5*\linewidth) $) -- cycle;
            }
          }{
            % Constant color, just a dot
            \draw[fill=\rightcolToUse] (\posxRot, \posyRot) circle(0.15);
          }
        }
      }
    }{}
