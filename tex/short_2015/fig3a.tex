%\begin{tikzpicture}[ scale=.25, show background rectangle]
\begin{tikzpicture}[scale=0.32, >=triangle 45, show background rectangle,
  declare function = {
    rotx(\x,\y,\th) = cos(\th)*\x - sin(\th)*\y; % NB! Cannot have space between parameters it seems!
    roty(\x,\y,\th) = sin(\th)*\x + cos(\th)*\y;
  }
  ]

  % View frustum
  \fill[fill=gray!20] (1, 0) -- (9, 0) -- (10, 5) -- (0, 5) -- cycle;
  % \clip (1, 0) -- (9, 0) -- (10, 5) -- (0, 5) -- cycle;
  % Clipping looks ok, but clipped content still expands bbox!! Solving this by reducing \pixels below from 13 to 11...


  % Defining some figure params

  \def \pixelwidth  {0.4}
  \def \pixelheight {0.3}
  \def \redstart    {2.2}
  \def \redpixels   {5}
  \def \pixels      {11}
  \def \linewidth   {1.25pt}
  \def \cx          {5}
  \def \cy          {3}
  \def \rightcol    {red!100}
  \def \leftcolL    {blue!100}
  %\def \leftcolR    {red!100!blue!0} % Why doesn't this work?! (The color becomes white)
  \def \leftcolR    {red!100}

  \pgfmathsetmacro\halfpixelwidth{0.5*\pixelwidth}


  \foreach \frame in {0, ..., 4} {

    \pgfmathsetmacro\theta{25*\frame}

    % The scene, using same coordinate calculations as in the loop below

    \ifthenelse{ \frame = 0 \OR \frame = 2 \OR \frame = 4} {
      \def \rightcolToUse {\rightcol}
      \def \leftcolLToUse {\leftcolL}
      \def \leftcolRToUse {\leftcolR}
    }{
      \def \rightcolToUse {red!10}
      \def \leftcolLToUse {blue!10}
      \def \leftcolRToUse {red!10}
    }

    \pgfmathsetmacro\s{1}
    \pgfmathsetmacro\posx{0.5*(\s-1) + 2}
    \pgfmathsetmacro\posy{3.5 - 0.6*(\s-1)}
    \pgfmathsetmacro\posxRot{rotx(\posx-\cx, \posy-\cy, \theta)+\cx}
    \pgfmathsetmacro\posyRot{roty(\posx-\cx, \posy-\cy, \theta)+\cy}
    \coordinate (left) at ( \posxRot, \posyRot );
    
    \pgfmathsetmacro\s{\redpixels - 1}
    \pgfmathsetmacro\posx{0.5*(\s-1) + 2}
    \pgfmathsetmacro\posy{3.5 - 0.6*(\s-1)}
    \pgfmathsetmacro\posxRot{rotx(\posx-\cx, \posy-\cy, \theta)+\cx}
    \pgfmathsetmacro\posyRot{roty(\posx-\cx, \posy-\cy, \theta)+\cy}
    \coordinate (middle) at ( \posxRot, \posyRot );
    
    % The offset necessary for the red and green lines to meet properly
    \pgfmathsetmacro\redgreenintersection{3.5 -(0.6+0.2)*(\s-1)}
    
    \pgfmathsetmacro\s{\pixels}
    \pgfmathsetmacro\posx{0.5*(\s-1) + 2}
    \pgfmathsetmacro\posy{0.2*(\s-1) + \redgreenintersection}
    \pgfmathsetmacro\posxRot{rotx(\posx-\cx, \posy-\cy, \theta)+\cx}
    \pgfmathsetmacro\posyRot{roty(\posx-\cx, \posy-\cy, \theta)+\cy}
    \coordinate (right) at ( \posxRot, \posyRot );
    
    % The "server-rendered model"
    \fill[left color=\leftcolLToUse, right color=\leftcolRToUse, draw=none]
      (left) -- (middle) -- +(0, 3*\linewidth) -- ($ (left) + (0, 3*\linewidth) $) -- cycle;
    % Why is the factor 3 needed here?!?!
    % And why is there a (very thin) border?
    % And why is it not equally thick everywhere?
    \draw[\rightcolToUse, line width=\linewidth] (middle) -- (right);

    % Showing splats
    \foreach \s in {1, ..., \pixels} {
      \ifthenelse{\s < \redpixels} {
        \pgfmathsetmacro\posx{0.5*(\s-1) + 2}
        \pgfmathsetmacro\posy{3.5 - 0.6*(\s-1)}
        \pgfmathsetmacro\posxRot{rotx(\posx-\cx, \posy-\cy, \theta)+\cx}
        \pgfmathsetmacro\posyRot{roty(\posx-\cx, \posy-\cy, \theta)+\cy}
        
        % Constant color
        \pgfmathsetmacro\t{100*\s/\redpixels}
        \pgfmathsetmacro\u{100*(1.0-\s/\redpixels)}
        %      \draw[red!\t!blue!\u, line width=1.5pt] (\posxRot, \posyRot) -- +(-\halfWidthToUse, 0);
        %      \draw[red!\t!blue!\u, line width=1.5pt] (\posxRot, \posyRot) -- +(\halfWidthToUse, 0);
        
%        \draw[fill=red!\t!blue!\u] (\posxRot, \posyRot) circle(0.15);
        
        %      \draw[red, fill=red!50] (\posxRot, \posyRot) circle(0.15);
      } {
        \pgfmathsetmacro\posx{0.5*(\s-1) + 2}
        \pgfmathsetmacro\posy{0.2*(\s-1) + \redgreenintersection}
        \pgfmathsetmacro\posxRot{rotx(\posx-\cx, \posy-\cy, \theta)+\cx}
        \pgfmathsetmacro\posyRot{roty(\posx-\cx, \posy-\cy, \theta)+\cy}

%        \draw[fill=\rightcolToUse] (\posxRot, \posyRot) circle(0.15);

      }
    }


  }


\end{tikzpicture}
